\documentclass[10pt,xcolor=dvipsnames,serif,professionalfont]{beamer} % " professionalfont " option
\usepackage[T1]{fontenc}
\usepackage{mathpazo}
\usepackage{eulervm}
\linespread{1.05}
\usepackage{xspace}
\usepackage{apacite}
\usepackage{rotating}
\usepackage{multirow}
\input{andrews_commands}
\setbeamerfont{title}{family=\it}
\setbeamerfont{frametitle}{family=\it}

\usepackage{tikz}
\usetikzlibrary{trees}

\usepackage{amssymb,latexsym,amsmath,amsfonts,amscd}

\usecolortheme[named=Gray]{structure} 
\setbeamercolor{titlelike}{fg=black!60!red}
\definecolor{Mygrey}{gray}{0.75}


\title[Probability]{Probability Basics}
\author[Andrews]{Mark Andrews}

\date{}

\DeclareSymbolFont{legacymaths}{OT1}{cmr}{m}{n}
\DeclareMathAccent{\dot}     {\mathalpha}{legacymaths}{95}
\DeclareMathAccent{\bar}     {\mathalpha}{legacymaths}{22}
\DeclareMathAccent{\tilde}     {\mathalpha}{legacymaths}{126}


\begin{document}
{
\begin{frame}
   \titlepage
\end{frame}
}
\begin{frame}
\frametitle{What is probability?}
\begin{itemize}
\item Probability is a means to quantify uncertainty.
\item If a variable can take on more than one value, probability can be used to describe the certainty that is will take each one of its possible values.
\item Probabilities must lie between zero and one.
\item The sum of the probabilities of all values of a variable must equal one.
\end{itemize}
\end{frame}

\begin{frame}
\frametitle{What is probability? Notation}
\begin{itemize}
\item If $X$ is a variable with possible values $\{x_1, x_2 \ldots x_k \ldots x_K\}$, then 
\begin{equation}
\Prob{X = x_k} 
\end{equation}
gives the probability that $X$ takes the value $x_k$.
\item The rules of probability require that
\begin{equation}
0 \leq \Prob{X = x_k} \leq 1 \quad \forall x_k 
\end{equation}
and
\begin{equation}
\sum_{k=1}^K \Prob{X = x_k} = 1.
\end{equation}

\end{itemize}
\end{frame}

\begin{frame}
	\frametitle{Joint probability}
\begin{itemize}
\item If $X$ is a variable with possible values $\{x_1, x_2 \ldots x_k \ldots x_K\}$, and $Y$ is a variable with possible values $\{y_1, y_2 \ldots y_k \ldots y_L\}$, then
\begin{equation}
\Prob{X = x_k, Y = y_l} 
\end{equation}
gives the probability that $X$ takes the value $x_k$ \emph{and} $Y$ takes the value of $y_l$.
\item The rules of probability require that
\begin{equation}
0 \leq \Prob{X = x_k, Y = y_l} \leq 1 \quad \forall x_k,y_l 
\end{equation}
and
\begin{equation}
	\sum_{l=1}^L \sum_{k=1}^K \Prob{X = x_k, Y = y_l} = 1.
\end{equation}

\end{itemize}
\end{frame}



\begin{frame}
\frametitle{Conditional, joint and marginal probability}
\begin{itemize}
\item Conditional probability is the probability distribution of a variable when the value of another variable is known, i.e. 
\begin{equation}
\Prob{X=x_k \given Y=y_l } 
\end{equation}
is read as the ``the probability that $X$ is $x_k$ \emph{given} $Y$ is $y_l$.
\item Conditional probability can be derived from the joint probability as follows:
\begin{equation}
\Prob{X=x_k \given Y=y_l } = \frac{\Prob{X=x_k, Y=y_l }}{ \Prob{Y=y_l } }
\end{equation}
\item Marginal probability can also be derived from the joint probability as follows:
\begin{equation}
\Prob{X=x_k} = \sum_{l=1}^L \Prob{X=x_k, Y=y_l }.
\end{equation}
\end{itemize}
\end{frame}



\section{Conditional, joint and marginal probability}

\begin{frame}
\frametitle{Conditional, joint and marginal probability}
\begin{itemize}
\item There at least three different types of probability distributions,
referred to as \emph{joint}, \emph{marginal} or \emph{conditional} probabilities. 
\item There are
fundamental relationships between them. 
\item We can illustrate these concepts by looking at survival rates of males and females on \emph{RMS Titanic}.
\end{itemize}
\end{frame}

\begin{frame}
\frametitle{Frequency distribution of survival rates on \emph{Titanic}}
\begin{itemize}
\item The following table shows the number of men and women who died or survived on the \emph{Titanic}:

\begin{center}
\begin{table}
\begin{tabular}{c|cc}
& Men & Women \\\hline
Perished & 1352 & 109 \\
Survived & 338 & 316 \\
\end{tabular}.
\end{table}
\end{center}
\item Altogether, there were 2115 onboard.
\end{itemize}
\end{frame}



\begin{frame}
\frametitle{From frequencies to probabilities}
\begin{itemize}
\item We can convert the frequencies 
\begin{center}
\begin{tabular}{c|cc}
& Men & Women \\\hline
Perished & 1352 & 109 \\
Survived & 338 & 316 \\
\end{tabular},
\end{center}
to the probabilities 
\begin{center}
\begin{tabular}{c|cc}
& Men & Women \\\hline
Perished & .64 & .05 \\
Survived & .16 & .15 \\
\end{tabular},
\end{center}
by dividing each frequency by the total number onboard, i.e. 2115.
\end{itemize}
\end{frame}

\begin{frame}
\frametitle{Joint probability tables}
\begin{itemize}
\item The table
\begin{center}
\begin{tabular}{c|cc}
& Men & Women \\\hline
Perished & .64 & .05 \\
Survived & .16 & .15 \\
\end{tabular},
\end{center}
is a \emph{joint probability} table.
\item It provides the probability for every combination of the two variables. 
\item It is just another probability distribution, like what we have met already.
\item Each element in the table lies between 0 and 1 and together they must sum to 1.
\end{itemize}
\end{frame}

\begin{frame}
\frametitle{Marginal probabilities from joint probabilities}
\begin{itemize}
\item From the table
{\scriptsize
\begin{center}
\begin{tabular}{c|cc}
& Men & Women \\\hline
Perished & .64 & .05 \\
Survived & .16 & .15 \\
\end{tabular},
\end{center}
}
what is the overall probability of dying?
\item Recall that each element in the table provides the probability for a combination of values of the two variables, e.g. the probability of dying and being male is $.64$.
\item Following the rules of probability, we calculate the probability of dying as follows:
\begin{align}
\Prob{\textrm{Perished}} &= \Prob{\textrm{Male \& Perished}} + \Prob{\textrm{Female \& Perished}},\\
&= .64 + .05 = .69
\end{align}
\end{itemize}
\end{frame}

\begin{frame}
\frametitle{Conditional probabilities from joint probabilities}
\begin{itemize}
\item From the table
{\scriptsize
\begin{center}
\begin{tabular}{c|cc}
& Men & Women \\\hline
Perished & .64 & .05 \\
Survived & .16 & .15 \\
\end{tabular},
\end{center}
}
what is the probability of dying if the person is a man?
\item First, the probability of being a man onboard \emph{Titanic} is 
\begin{align}
\Prob{\textrm{Male}} &= \Prob{\textrm{Male \& Perished}} + \Prob{\textrm{Male \& Survived}},\\
&= .64 + .16 = .8\ .
\end{align}
\end{itemize}
\end{frame}

\begin{frame}
\frametitle{Conditional probabilities from joint probabilities (cont'd)}
\begin{itemize}
\item Then, what is the probability of being a man and dying? This is simply $.64$.
\item Putting these together: 80\% of the ship's total was male and 64\% of the total were men who died. That means the fraction of men who died is 64/80, i.e.
\begin{align}
\Prob{\textrm{Perished} \given \textrm{Male}} &= \frac{\Prob{\textrm{Male \& Perished}}}{\Prob{\textrm{Male}}},\\
&= \frac{.64}{.8} = .8 
\end{align}
\end{itemize}
\end{frame}


\begin{frame}
\frametitle{Conditional, joint and marginal: More Rules}
\begin{itemize}
\item As we have
\begin{equation}
\Prob{X=x_k \given Y=y_l } = \frac{\Prob{X=x_k, Y=y_l }}{ \Prob{Y=y_l } }
\end{equation}
then we also have 
\begin{equation}
\Prob{X=x_k,Y=y_l} = \Prob{X=x_k\given Y=y_l}\Prob{Y=y_l},
\end{equation}
\item Likewise, given that
\begin{equation}
\Prob{Y=y_l \given X=x_k} = \frac{\Prob{Y=y_l,X=x_k}}{\Prob{X=x_k}}
\end{equation}
then 
\begin{equation}
\Prob{Y=y_l,X=x_k} = \Prob{Y=y_l \given X=x_k}\Prob{X=x_k}.
\end{equation}
\end{itemize}
\end{frame}

\begin{frame}
	\frametitle{Chain rule}
\begin{itemize}
	\item 
		\begin{align}
			\Prob{X, Y, Z} &= \Prob{X \given Y, Z} \Prob{Y\given Z} \Prob{Z},\\
			 &= \Prob{X \given Z, Y} \Prob{Z \given Y} \Prob{Y},\\
			 &= \Prob{Y \given X, Z} \Prob{X \given Z} \Prob{Z},\\
			 &= \ldots
		\end{align}
\end{itemize}
\end{frame}

\begin{frame}
	\frametitle{Independence and conditional independence}
\begin{itemize}
	\item If $\Prob{X, Y} = \Prob{X}\Prob{Y}$, then $X$ and $Y$ are independent.
	\item If $X$ and $Y$ are independent, their correlation is $0$.
	\item If $\Prob{X, Y \given Z} = \Prob{X \given Z} \Prob{Y \given Z}$, then $X$ and $Y$ are conditionally independent, conditional on $Z$.
	\item If $X$ and $Y$ are conditionally independent, conditional of $Z$, the parial correlation of $X$ and $Y$ given $Z$ is zero.
\end{itemize}
\end{frame}





\end{document}

